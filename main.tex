\documentclass[11pt,a4paper]{article}

\usepackage[margin=1in]{geometry}
\usepackage[utf8]{inputenc}
\usepackage[T1]{fontenc}
\usepackage{lmodern}
\usepackage{xcolor}
\usepackage{hyperref}
\usepackage{enumitem}
\usepackage{booktabs}
\usepackage{tabularx}
\usepackage{titlesec}
\usepackage{graphicx}
\usepackage{float}

\definecolor{accent}{HTML}{006666}
\hypersetup{
  colorlinks=true,
  linkcolor=accent,
  urlcolor=accent
}

\titleformat{\section}{\large\bfseries\color{accent}}{\thesection}{0.6em}{}
\titleformat{\subsection}{\bfseries}{\thesubsection}{0.5em}{}

\title{\vspace{-1em}\textbf{Project Documentation}\\
\large Personal Travel Assistant}
\author{Team: Group 1\\
Members: Simion Oltean Callian, Duica Sebastian, Denisa Sarafinceanu, Cosmin Savut}
\date{October 2025 – January 2026}

\begin{document}
\maketitle
\vspace{-1em}

\tableofcontents
\newpage

% =========================
\refstepcounter{section}
\section*{Phase 2 – Project Proposal Presentation}

\subsection{Description of the Project Problem Domain}

Travel planning entails transport, accommodation, and day activity booking all over
several online services. Most users alternate between booking sites, map apps, and calendars,
which enhance mental workload and likelihood of forgetting information; reduce steps and offering
clear feedback eliminates this task \cite{card1983psychology}. Our project — the \textbf{Personal Travel Assistant} — is
to compile all travel information into one easy interface to assist users in creating, editing, and
track their travels effectively.
Travel planning involves coordinating transport, accommodation, and daily activities across several online services.

\subsection{User Types}
\begin{itemize}
  \item \textbf{Students and young travelers} – seek affordable trips, rely mainly on mobile devices, moderate tech proficiency.
  \item \textbf{Business travelers} – need fast, reliable scheduling; high device proficiency; value reminders and sync with calendars.
  \item \textbf{Families or groups} – require clarity, simple navigation, and shared itineraries.
\end{itemize}

\subsubsection*{Knowledge Levels}
\begin{itemize}
  \item \textbf{Domain knowledge:} users can have only basic knowledge of bookings and travel choices.
  \item \textbf{Device knowledge:} is minimal; the interface uses familiar patterns (buttons, icons) and short texts, compatible with phones, tablets, and computers.
\end{itemize}

\subsubsection*{Special User Requirements}

The design follows \textit{WCAG 2.2 AA} accessibility goals and relies on human-centred design prin-
ciples to ensure equitable access \cite{iso9241}

\begin{itemize}
  \item High-contrast color theme and scalable text for visually impaired users.
  \item Voice-assisted search for users with motor disabilities.
  \item Touch-friendly layout and short forms to minimize effort.
\end{itemize}

\subsection{Main Challenges and Motivation}
\begin{itemize}
  \item \textbf{Fragmented tools} – travelers must manage several unrelated apps.
  \item \textbf{Uncertain conditions} – cancellations, weather, and delays require dynamic re-planning.
  \item \textbf{Information overload} – difficult to find trustworthy local experiences.
\end{itemize}
\textbf{Motivation:} integrating essential travel information in one accessible application saves time and reduces stress.
Usability research stresses the value of simplicity, visibility of system status, and error prevention when managing time-critical tasks \cite{nielsen1994usability}.

\subsection{Context of Use}
The app could be used:
\begin{itemize}
  \item On \textbf{mobile phones} during trips — noisy, interruptive environments (airports, buses), where concise feedback and large touch targets help maintain flow \cite{nielsen1994usability}.
  \item On \textbf{desktops} for planning — quieter, analytical context suited to comparison and review.
  \item With \textbf{time-sensitive tasks} (boarding, check-ins), so quick feedback and reminders are essential.
  \item \textbf{Offline access} is needed for tickets and maps when connectivity is lost.
\end{itemize}

\subsection{Existing Applications}
\begin{tabularx}{\linewidth}{l X X}
\toprule
\textbf{App} & \textbf{Advantages} & \textbf{Limitations} \\
\midrule
TripAdvisor & Many reviews and location data & Hard to personalize; no schedule integration \\
Booking.com & Reliable booking engine & Limited itinerary organization \\
Google Maps & Excellent navigation support & Overwhelming interface for trip planning \\
TripIt & Auto-import of bookings & Premium features locked; limited local discovery \\
\bottomrule
\end{tabularx}

\subsection{Interconnection with Other Tools}
The Personal Travel Assistant integrates existing services to avoid manual re-entry of data:
\begin{itemize}
  \item \textbf{Google Maps API} – display routes and travel times.
  \item \textbf{Email parsing} – detect booking confirmations and import details.
\end{itemize}
Consistent interaction patterns and clearly scoped integrations support efficiency and satisfaction in line with human-centred design guidance \cite{iso9241}.

% =========================
\newpage
\refstepcounter{section}
\section*{Phase 2a – Task Analysis Examples}

This phase presents two simple travel related tasks that typical users of the Personal Travel Assistant project would perform.

\subsection{Task 1 — Plan a Weekend Trip}
\textbf{Goal:} create a basic travel plan for a short trip.

\begin{enumerate}
  \item Open the app and tap “Plan new trip.”
  \item Enter the destination and dates.
  \item Select one suggested hotel.
  \item Choose one transport option (train or flight).
  \item Save the trip.
\end{enumerate}

\textbf{Expected result:} the trip is saved, and the user can see it on the main screen with hotel and travel details.

\subsection{Task 2 — Change the Travel Date}
\textbf{Goal:} adjust a planned trip when the departure date changes.

\begin{enumerate}
  \item Open “My Trips” and select the saved trip.
  \item Tap “Edit” and choose a new start date.
  \item The app updates the return date automatically.
  \item Confirm and save the changes.
\end{enumerate}

\textbf{Expected result:} the trip shows the new dates, and reminders update automatically.

% =========================
\newpage
\refstepcounter{section}
\section*{Phase 3 – Task Analysis}

This phase describes 12 simple and realistic tasks that users can perform with the \textbf{Personal Travel Assistant}.  
Each task includes the starting point, type of user, actions performed, motivation, and the context in which it occurs.

\subsection*{Task 1 – Explore Different Destinations}
\textbf{Starting point:} Landing page \\
\textbf{Users:} All users looking for travel inspiration \\
\textbf{What:} The user browses the landing page to see featured destinations and travel options. They navigate to the Trip Planner to enter a destination name and see the form fields for dates, travelers, trip type, and budget. They can then explore related tools like Weather to check conditions for their chosen destination, and Activities to see what experiences might be available. \\
\textbf{Why:} To get familiar with the app's capabilities and start thinking about potential travel destinations. \\
\textbf{Context:} Done casually at home while browsing for travel ideas, typically on mobile or desktop during free time.

\subsection*{Task 2 – Check Weather and Plan Accordingly}
\textbf{Starting point:} Weather component \\
\textbf{Users:} Travelers planning outfit and activity choices \\
\textbf{What:} The user navigates to the Weather section to check current conditions and forecast information for their destination. They review temperature ranges, precipitation chances, and weather patterns. Based on this information, they navigate to the Wardrobe Planner to see clothing recommendations for different weather scenarios (beach, winter, business, adventure) and adjust their packing plans accordingly. \\
\textbf{Why:} To make informed decisions about what to pack and which activities might be suitable for the expected weather conditions. \\
\textbf{Context:} Done during trip planning phase when deciding on clothing and activities, typically at home using desktop or mobile for quick checks.

\subsection*{Task 3 – Plan Trip Details Using the Trip Planner}
\textbf{Starting point:} Trip Planner interface \\
\textbf{Users:} Users ready to organize their travel details \\
\textbf{What:} The user fills out the trip planning form by entering destination information, selecting start and end dates, specifying the number of travelers, choosing a trip type (leisure, business, adventure, cultural, relaxation), and setting a preliminary budget. They explore the action cards to navigate to different planning tools like accommodation search, activities discovery, weather checking, currency conversion, and budget calculation to get ideas for their trip structure. \\
\textbf{Why:} To organize basic trip information and explore the different planning tools available in the application. \\
\textbf{Context:} Done when ready to move from general browsing to specific trip planning, typically at home using desktop for form completion and exploration.

\subsection*{Task 4 – Research and Browse Accommodations}
\textbf{Starting point:} Accommodation search interface \\
\textbf{Users:} Travelers looking for places to stay \\
\textbf{What:} The user explores the accommodation section to browse different types of lodging options including hotels, resorts, and lodges. They review accommodation details like pricing, amenities (WiFi, pool, breakfast, parking, restaurant, spa), ratings, and view images of different properties. They can filter options and compare different accommodation types to understand what's available in their budget range. \\
\textbf{Why:} To explore accommodation options and understand pricing and amenities to help with booking decisions later. \\
\textbf{Context:} Done during the planning phase when considering where to stay, typically at home using desktop for detailed comparison of options and amenities.

\subsection*{Task 5 – Discover and Explore Activities}
\textbf{Starting point:} Activities browser interface \\
\textbf{Users:} Travelers seeking experiences and things to do \\
\textbf{What:} The user browses the activities section using category tabs (all, cultural, food, adventure) to discover different types of experiences. They review activity details including descriptions, duration, pricing, participant limits, ratings, and location information. They can view images and read about various activities like cultural tours, cooking classes, adventure sports, and local experiences to get ideas for their itinerary. \\
\textbf{Why:} To discover interesting activities and experiences available at their destination and understand pricing and time commitments for different types of activities. \\
\textbf{Context:} Done during planning when looking for things to do, often at home browsing for inspiration or specific activity types, using mobile or desktop for exploration.

\subsection*{Task 6 – Plan Wardrobe Using Weather Information}
\textbf{Starting point:} Wardrobe Planner interface \\
\textbf{Users:} Travelers wanting to pack appropriately \\
\textbf{What:} The user uses the Wardrobe Planner to explore different packing scenarios based on trip type and weather conditions. They can select from pre-defined wardrobe categories like Beach/Summer, Business Travel, Adventure/Outdoor, or Winter Travel to see recommended clothing items with quantities and essential status. They review suggested items and their importance levels to help create their own packing checklist based on their destination's weather and planned activities. \\
\textbf{Why:} To understand what clothing to pack for different types of trips and weather conditions, ensuring they bring appropriate items. \\
\textbf{Context:} Done during packing preparation, typically at home a few days before departure when deciding what clothes to bring, using mobile or desktop for reference.

\subsection*{Task 7 – Create and Organize Packing Lists}
\textbf{Starting point:} Packing List interface \\
\textbf{Users:} Organized travelers preparing for departure \\
\textbf{What:} The user explores the Packing List tool to see different categories of items they might need for their trip. They review suggested items organized by category (clothing, electronics, toiletries, documents, etc.) and can use this as a reference to create their own packing checklist. They can see recommendations for different trip types and use the interface to help ensure they don't forget essential items. \\
\textbf{Why:} To organize packing efficiently and ensure all necessary items are remembered for the trip. \\
\textbf{Context:} Done in the days leading up to departure when preparing to pack, typically at home using the tool as a reference while gathering items and organizing luggage.

\subsection*{Task 8 – View and Plan Daily Itinerary}
\textbf{Starting point:} Itinerary planning interface \\
\textbf{Users:} Travelers organizing their daily schedules \\
\textbf{What:} The user explores the Itinerary section to see how daily schedules can be organized. They review example itineraries with different activities planned throughout the day, including meal times, sightseeing, activities, and leisure time. They can see how activities are organized by time, location, and type (food, culture, leisure, adventure) to understand how to structure their own daily plans. \\
\textbf{Why:} To understand how to organize daily schedules effectively and see examples of well-structured itineraries for planning their own days. \\
\textbf{Context:} Done when planning daily activities and schedules, typically at home when organizing the flow of their trip, using desktop for detailed planning or mobile for quick reference.

\subsection*{Task 9 – Calculate and Track Travel Budget}
\textbf{Starting point:} Budget Calculator interface \\
\textbf{Users:} Budget-conscious travelers and trip planners \\
\textbf{What:} The user explores the Budget Calculator to understand travel costs and manage expenses. They review different budget templates (Budget Traveler, Mid-Range Explorer, Luxury Traveler) to see example cost breakdowns across categories like flights, accommodation, food, transportation, activities, and shopping. They can explore the different tabs (Overview, Expenses, Analytics, Templates) to understand how to track spending and manage budgets for different types of trips. \\
\textbf{Why:} To understand typical travel costs and learn how to organize and track expenses for their trip planning. \\
\textbf{Context:} Done during early planning when setting budget expectations or when trying to estimate total trip costs, typically at home using desktop for detailed budget analysis.

\subsection*{Task 10 – Convert Currency and Understand Exchange Rates}
\textbf{Starting point:} Currency Converter interface \\
\textbf{Users:} International travelers dealing with foreign currencies \\
\textbf{What:} The user accesses the Currency Converter to understand exchange rates between their home currency and their destination currency. They can input amounts to see conversions and get familiar with the approximate costs of items in their destination country. This helps them understand the value of money in different currencies and plan their spending accordingly. \\
\textbf{Why:} To understand currency exchange rates and get familiar with costs in their destination country for better budget planning and spending decisions. \\
\textbf{Context:} Done during planning when trying to understand destination costs, or during travel when needing quick currency conversions, typically using mobile for convenience or desktop during planning sessions.

\subsection*{Task 11 – Access Emergency Information and Contacts}
\textbf{Starting point:} Emergency Contacts interface \\
\textbf{Users:} Safety-conscious travelers preparing for potential emergencies \\
\textbf{What:} The user explores the Emergency Contacts section to understand what safety information and emergency contacts they should have available during travel. They review categories of important contacts and information they might need, such as local emergency services, embassy contacts, medical facilities, and important phone numbers. This helps them prepare their own emergency contact list and safety information. \\
\textbf{Why:} To understand what emergency information to gather and have available for safe travel planning and preparation. \\
\textbf{Context:} Done during trip preparation when organizing safety information, typically at home when collecting important contacts and numbers to have available during travel, using desktop or mobile for reference.

\subsection*{Task 12 – Navigate Between Different Planning Tools}
\textbf{Starting point:} Any component interface \\
\textbf{Users:} All users exploring the application's features \\
\textbf{What:} The user practices navigating between different sections of the travel planning app using the navigation system. They move between the landing page, trip planner, accommodation browser, activities explorer, wardrobe planner, itinerary viewer, weather checker, currency converter, budget calculator, packing list, and emergency contacts. They explore how each tool connects to others and understand the overall workflow of trip planning using the app. \\
\textbf{Why:} To become familiar with the app's structure and understand how different planning tools work together for comprehensive trip planning. \\
\textbf{Context:} Done when first exploring the app or when wanting to understand all available features, typically at home using desktop or mobile to get familiar with the interface and navigation flow.

% =========================
\newpage
\refstepcounter{section}
\section*{Phase 5 – User Scenarios Based on Task-Centered Design}

This phase applies Task-Centered User Interface Design principles (Lewis \& Rieman, 1993) to generate detailed user scenarios for selected tasks. Following Section 2.3 "Using the Tasks in Design," each scenario demonstrates how users accomplish real goals through specific interface interactions, grounded in user motivation and validated design patterns.

\subsection*{Task 1: Explore Different Destinations}

\textbf{Design-independent task goal:}
The user wants to get familiar with the app's capabilities and start thinking about potential travel destinations.

\subsubsection*{Scenario A (Design-specific)}

\textbf{Starting point:} User arrives at the landing page after opening the app.

\textbf{Step-by-step user actions and system responses:}
\begin{enumerate}
\item User sees the landing page with hero section, feature cards, and navigation options
\item User scrolls through the feature cards showing "Smart Planning," "Local Experiences," and other capabilities
\item User clicks the "Start Your Journey" button — system navigates to the Trip Planner interface
\item User sees the Trip Planner form with destination input field and action cards for different planning tools
\item User explores the action cards by hovering over them — system shows hover effects and descriptions
\item User clicks on "Check Weather" action card — system navigates to Weather component
\item User returns to Trip Planner using navigation menu — system shows the form interface again
\end{enumerate}

\textbf{Success outcome:} User understands the app's structure and available planning tools, gaining familiarity with the interface for future use.

\textbf{Error or exception case:} If navigation fails, system could show a loading state. If user clicks invalid links, system remains on current page with no action.

\textbf{Design rationale:} The landing page serves as an effective entry point by showcasing capabilities without overwhelming users. The action cards provide clear pathways to different tools, supporting the task-centered principle of making system capabilities visible and accessible.

\subsubsection*{Scenario B (Alternative successful flow)}

\textbf{Starting point:} User navigates directly to Trip Planner from the navigation menu.

\textbf{Alternative flow:} User begins by exploring the navigation menu to see all available sections, then visits multiple components (Weather, Activities, Accommodation) before returning to Trip Planner. This approach provides broader context about available tools before focusing on trip planning.

\subsection*{Task 2: Check Weather and Plan Accordingly}

\textbf{Design-independent task goal:}
The user wants to understand weather conditions at their destination to make informed decisions about clothing and activities.

\subsubsection*{Scenario A (Design-specific)}

\textbf{Starting point:} User navigates to the Weather component from Trip Planner action cards.

\textbf{Step-by-step user actions and system responses:}
\begin{enumerate}
\item User sees Weather interface with location input field and weather display tabs
\item User clicks in the location search field — system shows placeholder text "Enter city name"
\item User types "Paris" — system shows mock weather data for Paris with current conditions
\item User clicks "Forecast" tab — system displays 7-day weather forecast with temperature ranges and conditions
\item User reviews forecast showing temperatures, precipitation chances, and weather icons
\item User clicks "Alerts" tab — system shows weather alerts and recommendations
\item User navigates to Wardrobe Planner — system shows weather-based packing scenarios
\end{enumerate}

\textbf{Success outcome:} User understands weather patterns and can make informed decisions about packing and activity choices based on the displayed forecast information.

\textbf{Error or exception case:} If location search fails, system could show default weather data. If forecast data is unavailable, system displays general weather patterns.

\textbf{Design rationale:} The tabbed interface organizes weather information logically, supporting users' natural progression from current conditions to planning implications. Mock data provides realistic examples without requiring external API integration.

\subsubsection*{Scenario B (Alternative successful flow)}

\textbf{Starting point:} User accesses Weather from the main navigation menu.

\textbf{Alternative flow:} User first explores different weather tabs to understand the interface structure, then searches for their specific destination. This methodical approach helps users understand all available weather information before focusing on their location.

\subsection*{Task 3: Plan Trip Details Using the Trip Planner}

\textbf{Design-independent task goal:}
The user wants to organize and document their travel plans by entering destination information, dates, traveler details, and trip preferences to create a foundation for detailed trip planning.

\subsubsection*{Scenario A (Design-specific)}

\textbf{Starting point:} User arrives at the Trip Planner interface after clicking "Start Your Journey" on the landing page.

\textbf{Step-by-step user actions and system responses:}
\begin{enumerate}
\item User sees the Trip Planner header with "Trip Planning" badge and form titled "Trip Details"
\item User clicks in the "Destination" input field (with MapPin icon) and types "Paris, France" — system shows text appearing in real-time with placeholder text clearing
\item User clicks "Start Date" date picker field — system opens native date selector, user selects March 15, 2026
\item User clicks "End Date" date picker field — system opens date selector, user selects March 22, 2026
\item User clicks in "Number of Travelers" field (with Users icon) and enters "2" — system accepts numeric input
\item User clicks "Trip Type" dropdown — system reveals five options (Leisure, Business, Adventure, Cultural, Relaxation), user selects "Cultural"
\item User clicks in "Budget (USD)" field and enters "3500" — system displays the amount
\item User reviews the form and scrolls down to see action cards for next planning steps
\end{enumerate}

\textbf{Success outcome:} Form displays all entered information clearly, and user can proceed to explore accommodation, activities, or other planning tools via the action cards grid.

\textbf{Error or exception case:} If user enters invalid dates (end date before start date), system could highlight the error with red border and helper text. If budget is non-numeric, system shows validation message.

\textbf{Design rationale:} This flow follows task-centered principles by focusing on the user's immediate goal of documenting trip basics. The form uses familiar input patterns (date pickers, dropdowns) that reduce cognitive load. Visual icons (MapPin, Users) provide immediate context, and the logical top-to-bottom flow matches how people mentally organize trip information: destination first, then dates, then details.

\subsubsection*{Scenario B (Alternative successful flow)}

\textbf{Starting point:} User navigates directly to Trip Planner from the navigation menu.

\textbf{Alternative flow:} User begins by exploring the action cards first to understand available tools, then returns to fill out the planning form with more informed choices about trip type and budget based on seeing accommodation and activity options. User fills out form in different order (dates first, then destination based on seasonal considerations), achieving the same goal through a more exploratory approach.

\subsection*{Task 4: Research and Browse Accommodations}

\textbf{Design-independent task goal:}
The user wants to explore accommodation options and understand pricing and amenities to help with booking decisions later.

\subsubsection*{Scenario A (Design-specific)}

\textbf{Starting point:} User clicks "Find Accommodation" action card from Trip Planner.

\textbf{Step-by-step user actions and system responses:}
\begin{enumerate}
\item User sees Accommodation interface with accommodation type cards and search options
\item User browses through accommodation cards showing hotels, resorts, and lodges
\item User clicks on a hotel card — system expands to show detailed amenities list (WiFi, pool, breakfast, parking)
\item User reviews pricing information and ratings displayed on each card
\item User clicks on different accommodation types — system shows varying amenities and price ranges
\item User compares accommodation options by scrolling through the grid layout
\item User navigates back to Trip Planner — system returns to planning interface
\end{enumerate}

\textbf{Success outcome:} User understands different accommodation options, pricing ranges, and amenities available for their destination.

\textbf{Error or exception case:} If accommodation images fail to load, system shows placeholder images. If no accommodations match criteria, system displays all available options.

\textbf{Design rationale:} Card-based layout enables efficient comparison of multiple options simultaneously. Detailed amenity lists support informed decision-making about accommodation choices.

\subsubsection*{Scenario B (Alternative successful flow)}

\textbf{Starting point:} User accesses Accommodation from the main navigation menu.

\textbf{Alternative flow:} User first explores all accommodation types to understand the range of options, then focuses on specific amenities that matter most for their trip type. This comprehensive approach helps users make more informed accommodation decisions.

\subsection*{Task 5: Discover and Explore Activities}

\textbf{Design-independent task goal:}
The user wants to discover interesting activities and experiences available at their destination to understand options, pricing, and time commitments for different types of activities.

\subsubsection*{Scenario A (Design-specific)}

\textbf{Starting point:} User clicks "Discover Activities" action card from Trip Planner or navigates directly to Activities section.

\textbf{Step-by-step user actions and system responses:}
\begin{enumerate}
\item User sees Activities header with "Activity Finder" badge and category tabs (All, Cultural, Food, Adventure)
\item User clicks "Cultural" tab — system filters activities and displays only cultural experiences with smooth transition animation
\item User scrolls through cultural activity cards, each showing image, title, location, duration, price, rating, and participant count
\item User clicks on "Temple Tour with Archaeologist" activity card — system highlights the card and shows detailed description expanding
\item User reviews activity details: "Explore ancient ruins with expert archaeologist," 4 hours duration, \$45 per person, 4.9 rating
\item User clicks "Adventure" tab — system transitions to show adventure activities like "Sunrise Mountain Hike"
\item User clicks on mountain hike card — system shows details: 5 hours, \$80, 4.8 rating, up to 15 people
\item User navigates back to "All" tab to compare across categories
\end{enumerate}

\textbf{Success outcome:} User has discovered multiple activity options across categories, understands pricing structure, and can mentally plan their trip activities and budget.

\textbf{Error or exception case:} If no activities exist for a category, system shows empty state with message "No activities found" and suggestion to try other categories. If images fail to load, system shows placeholder with activity title.

\textbf{Design rationale:} The tabbed interface supports the user's natural browsing behavior by allowing quick category switching without losing context. Card-based layout enables efficient scanning of multiple options simultaneously. Key information (price, duration, rating) is prominently displayed for quick comparison, supporting the task-centered principle of immediate access to decision-critical information.

\subsubsection*{Scenario B (Alternative successful flow)}

\textbf{Alternative flow:} User begins with "All" tab to get overview of available activity types, then uses visual cues (activity images and titles) to identify interests before switching to specific category tabs. User discovery process is more visual-first rather than category-first, but achieves same understanding of available options.

\subsection*{Task 6: Plan Wardrobe Using Weather Information}

\textbf{Design-independent task goal:}
The user wants to understand what clothing to pack for different types of trips and weather conditions, ensuring they bring appropriate items for their destination and planned activities.

\subsubsection*{Scenario A (Design-specific)}

\textbf{Starting point:} User navigates to Wardrobe Planner from Trip Planner action cards or main navigation.

\textbf{Step-by-step user actions and system responses:}
\begin{enumerate}
\item User sees Wardrobe Planner interface with weather-based packing scenarios
\item User clicks "Beach/Summer" scenario button — system displays summer clothing checklist with items like "Swimwear (2-3 pieces)", "Sundresses (3-4)", "Sandals (2 pairs)"
\item System shows each item with quantity recommendation and essential status (marked with star icon for must-have items)
\item User reviews essential items highlighted in the list: sunscreen, hat, sunglasses marked as critical
\item User clicks "Adventure/Outdoor" scenario — system transitions to show outdoor gear list including "Hiking boots", "Waterproof jacket", "Base layers"
\item System displays quantity recommendations and marks essential items for safety and comfort
\item User compares scenarios to understand different packing needs for various trip types
\item User can reference this information while packing or planning their wardrobe
\end{enumerate}

\textbf{Success outcome:} User understands what clothing types and quantities to pack for their specific trip weather and activity type, with clear guidance on essential vs. optional items.

\textbf{Error or exception case:} If user's trip spans multiple weather conditions, system could suggest checking multiple scenarios. If seasonal weather data is unavailable, system defaults to general recommendations with appropriate disclaimers.

\textbf{Design rationale:} Scenario-based organization matches how people think about packing — by context and weather conditions. The essential item marking addresses the core user need of not forgetting critical items. Quantity recommendations reduce decision-making overhead, following task-centered design principles of supporting user expertise rather than requiring it.

\subsubsection*{Scenario B (Alternative successful flow)}

\textbf{Alternative flow:} User first visits Weather component to check their destination's forecast, then navigates to Wardrobe Planner with specific weather expectations. This creates a more informed wardrobe selection process where the user cross-references actual weather data with clothing scenarios for more precise packing decisions.

\subsection*{Task 7: Create and Organize Packing Lists}

\textbf{Design-independent task goal:}
The user wants to organize packing efficiently and ensure all necessary items are remembered for the trip.

\subsubsection*{Scenario A (Design-specific)}

\textbf{Starting point:} User navigates to Packing List from Trip Planner action cards.

\textbf{Step-by-step user actions and system responses:}
\begin{enumerate}
\item User sees Packing List interface with category tabs and item checklists
\item User clicks "Clothing" tab — system displays clothing items with checkboxes and quantity recommendations
\item User checks off items they plan to pack — system shows checkmarks and updates progress indicators
\item User clicks "Electronics" tab — system shows electronics checklist with essential items like phone charger, adapter
\item User reviews different category tabs (Toiletries, Documents, Miscellaneous) — system shows organized item lists for each
\item User sees progress indicators showing completion percentage for each category
\item User can use this as a reference while physically packing their bags
\end{enumerate}

\textbf{Success outcome:} User has a comprehensive checklist of items to pack, organized by category, with progress tracking to ensure nothing is forgotten.

\textbf{Error or exception case:} If user tries to check items that don't exist, system maintains current state. If progress calculation fails, system shows basic completion indicators.

\textbf{Design rationale:} Category-based organization matches how people naturally think about packing. Checkboxes and progress indicators provide clear feedback on packing completion, supporting the task-centered principle of making progress visible.

\subsubsection*{Scenario B (Alternative successful flow)}

\textbf{Starting point:} User accesses Packing List from main navigation.

\textbf{Alternative flow:} User first reviews all categories to get an overview of required items, then systematically works through each category checking items. This comprehensive approach ensures thorough packing preparation.

\subsection*{Task 8: View and Plan Daily Itinerary}

\textbf{Design-independent task goal:}
The user wants to understand how to organize daily schedules effectively and see examples of well-structured itineraries for planning their own days.

\subsubsection*{Scenario A (Design-specific)}

\textbf{Starting point:} User navigates to Itinerary from Trip Planner action cards.

\textbf{Step-by-step user actions and system responses:}
\begin{enumerate}
\item User sees Itinerary interface with day navigation and activity timeline
\item User clicks on "Day 1" — system displays scheduled activities with times, locations, and types
\item User reviews activity details showing breakfast at hotel, city walking tour, lunch at market, museum visit, sunset at beach
\item User clicks on different days — system shows varying activity schedules and meal times
\item User sees activity types color-coded (food, culture, leisure, adventure) with icons
\item User can mentally plan their own daily schedules based on the example structure
\item User navigates back to Trip Planner — system returns to planning interface
\end{enumerate}

\textbf{Success outcome:} User understands how to structure daily itineraries with balanced activities, meals, and leisure time.

\textbf{Error or exception case:} If day navigation fails, system shows default day. If activity details are missing, system displays basic activity information.

\textbf{Design rationale:} Timeline-based layout matches how people think about daily schedules. Color-coding and icons make activity types immediately recognizable, supporting efficient itinerary planning.

\subsubsection*{Scenario B (Alternative successful flow)}

\textbf{Starting point:} User accesses Itinerary from main navigation.

\textbf{Alternative flow:} User explores all days first to understand the overall trip structure, then focuses on specific days for detailed planning. This approach provides context before diving into individual day planning.

\subsection*{Task 9: Calculate and Track Travel Budget}

\textbf{Design-independent task goal:}
The user wants to understand typical travel costs and learn how to organize and track expenses for their trip planning and budget management.

\subsubsection*{Scenario A (Design-specific)}

\textbf{Starting point:} User accesses Budget Calculator from Trip Planner action cards or navigation menu.

\textbf{Step-by-step user actions and system responses:}
\begin{enumerate}
\item User sees Budget Calculator interface with tabs: Overview, Expenses, Analytics, Templates
\item User clicks "Templates" tab — system displays three budget template cards: Budget Traveler (\$1,350), Mid-Range Explorer (\$2,650), Luxury Traveler
\item User clicks "Mid-Range Explorer" template card — system expands to show category breakdown: Flights \$500, Accommodation \$800, Food \& Dining \$400, etc.
\item System displays each category with amount and visual progress bar showing proportional spending allocation
\item User clicks "Budget Traveler" to compare — system shows lower-cost breakdown with different allocation ratios
\item User clicks "Overview" tab — system shows budget tracking interface with categories and current spending status
\item User reviews expense categories and understands typical cost distribution for different travel styles
\item User can use this information to set realistic budget expectations for their trip type
\end{enumerate}

\textbf{Success outcome:} User understands typical cost ranges for different travel styles and how expenses are typically distributed across categories, enabling informed budget planning.

\textbf{Error or exception case:} If template data fails to load, system shows basic category list with placeholder amounts and error message. If user tries to access advanced features without data, system guides them to start with templates.

\textbf{Design rationale:} Template-first approach reduces the overwhelming nature of budget planning by providing realistic starting points. Category-based organization matches how people naturally think about travel expenses. Visual representation through progress bars and clear totals supports quick comprehension and comparison, following task-centered principles of reducing cognitive load while supporting decision-making.

\subsubsection*{Scenario B (Alternative successful flow)}

\textbf{Alternative flow:} User starts with "Overview" tab to understand the budget tracking interface, then navigates to "Templates" to see realistic examples. This approach suits users who prefer to understand the tool's capabilities before exploring content, achieving the same budget understanding through a more methodology-focused exploration.

\subsection*{Task 10: Convert Currency and Understand Exchange Rates}

\textbf{Design-independent task goal:}
The user wants to understand currency exchange rates and get familiar with costs in their destination country for better budget planning and spending decisions.

\subsubsection*{Scenario A (Design-specific)}

\textbf{Starting point:} User navigates to Currency Converter from Trip Planner action cards.

\textbf{Step-by-step user actions and system responses:}
\begin{enumerate}
\item User sees Currency Converter interface with currency selection dropdowns and conversion display
\item User clicks "From" currency dropdown — system shows list of available currencies with flags and codes
\item User selects "USD" — system updates the interface
\item User clicks "To" currency dropdown — system shows destination currencies
\item User selects "EUR" for European destination — system shows current exchange rate
\item User enters amount "1000" in USD field — system calculates and displays equivalent in EUR
\item User sees conversion result with current exchange rate information
\item User can use this to understand approximate costs in destination currency
\end{enumerate}

\textbf{Success outcome:} User understands currency conversion rates and can estimate costs in their destination currency for budget planning.

\textbf{Error or exception case:} If currency selection fails, system shows default currencies. If conversion calculation errors occur, system displays last known rates.

\textbf{Design rationale:} Simple input-output interface matches users' mental model of currency conversion. Clear display of exchange rates supports informed financial planning decisions.

\subsubsection*{Scenario B (Alternative successful flow)}

\textbf{Starting point:} User accesses Currency Converter from main navigation.

\textbf{Alternative flow:} User explores different currency pairs first to understand exchange rate variations, then focuses on their specific travel currencies. This exploratory approach helps users understand broader currency relationships.

\subsection*{Task 11: Access Emergency Information and Contacts}

\textbf{Design-independent task goal:}
The user wants to understand what emergency information to gather and have available for safe travel planning and preparation.

\subsubsection*{Scenario A (Design-specific)}

\textbf{Starting point:} User navigates to Emergency Contacts from Trip Planner action cards.

\textbf{Step-by-step user actions and system responses:}
\begin{enumerate}
\item User sees Emergency Contacts interface with contact categories and emergency services
\item User reviews emergency service cards showing phone numbers for police, fire, medical services
\item User clicks "Add Contact" button — system shows form for adding personal emergency contacts
\item User fills in contact details: name, relationship, phone number, email — system saves the contact
\item User sees contact organized by category (family, medical, travel)
\item User reviews emergency service information and important phone numbers
\item User can reference this information for safety planning
\end{enumerate}

\textbf{Success outcome:} User has access to important emergency contacts and services information for their destination.

\textbf{Error or exception case:} If contact saving fails, system shows error message. If emergency service data is unavailable, system displays general emergency guidelines.

\textbf{Design rationale:} Category-based organization makes emergency information easily accessible when needed. Clear contact management supports proactive safety planning.

\subsubsection*{Scenario B (Alternative successful flow)}

\textbf{Starting point:} User accesses Emergency Contacts from main navigation.

\textbf{Alternative flow:} User first reviews all emergency services to understand available resources, then adds personal contacts. This comprehensive approach ensures complete emergency preparedness.

\subsection*{Task 12: Navigate Between Different Planning Tools}

\textbf{Design-independent task goal:}
The user wants to become familiar with the app's structure and understand how different planning tools work together for comprehensive trip planning.

\subsubsection*{Scenario A (Design-specific)}

\textbf{Starting point:} User starts at the landing page.

\textbf{Step-by-step user actions and system responses:}
\begin{enumerate}
\item User sees landing page with navigation menu showing all available sections
\item User clicks navigation menu — system expands to show Trip Planner, Accommodation, Activities, Weather, and other tools
\item User clicks "Trip Planner" — system navigates to planning interface
\item User clicks navigation again — system shows menu, user selects "Activities"
\item User explores Activities interface, then uses navigation to visit "Weather"
\item User continues navigating between different sections — system smoothly transitions between interfaces
\item User returns to landing page — system shows overview of all planning capabilities
\end{enumerate}

\textbf{Success outcome:} User understands the app's navigation structure and can efficiently move between different planning tools as needed.

\textbf{Error or exception case:} If navigation fails, system shows current page. If invalid navigation occurs, system remains stable on current interface.

\textbf{Design rationale:} Consistent navigation menu provides predictable access to all tools. Smooth transitions between sections support efficient exploration and task switching.

\subsubsection*{Scenario B (Alternative successful flow)}

\textbf{Starting point:} User starts at Trip Planner.

\textbf{Alternative flow:} User explores action cards within Trip Planner to navigate to different tools, then uses the navigation menu for broader exploration. This hybrid approach combines direct tool access with comprehensive navigation understanding.


\newpage
\refstepcounter{section}
\section*{Phase 6 – Cognitive Walkthrough Evaluation Report}

This phase presents a cognitive walkthrough evaluation of the Personal Travel Assistant interface prototype, following the methodology described by Wharton et al. (1994). The evaluation examines 8 user tasks from Phase 3, stepping through the scenarios defined in Phase 5. At each step, we analyze five critical questions to identify potential usability issues and validate design decisions.

\subsection*{Task 1: Plan Trip Details Using the Trip Planner}
\begin{figure}[H]
    \centering
    \includegraphics[width=\textwidth]{images/home-page.png}
    \caption{Home page}
\end{figure}

\begin{figure}[H]
    \centering
    \includegraphics[width=\textwidth]{images/trip-form.png}
    \caption{Trip Planner page}
\end{figure}

\textbf{Starting point:} The user accesses the landing page or navigates to the Trip Planner. The interface clearly presents the main entry points for trip planning, including buttons like "Start Your Journey" and visible action cards for accommodation and activities.

\textbf{Action: User begins entering trip details}  
The user fills in the form fields (Destination, Start Date, End Date, etc.). As they type or select values, the system clears placeholders and immediately shows their input, providing real-time feedback. Icons and labels clearly indicate the purpose of each field.

\textbf{Cognitive Walkthrough Questions:}
\begin{enumerate}
    \item \textbf{Will the user be trying to produce the effect?} - Yes, the user’s goal is to start planning their trip, and the “Trip Details” heading clearly signals that filling in the form is the correct first step.
    \item \textbf{Will the user see the correct control?} - Yes, the input fields (Destination with MapPin icon, Start and End Dates with calendar icons, etc.) are prominent at the top of the page, making the intended controls highly visible.
    \item \textbf{Will the user see that the control produces the desired effect?} - Yes, standard affordances (text boxes, date pickers, dropdowns) make the effect of interacting with each field immediately clear.
    \item \textbf{Is there another control that the user might select instead of the correct one?} - Possibly, the action cards on the right are visible, but they serve different purposes and are visually secondary, so confusion is unlikely.
    \item \textbf{Will the user understand the feedback to proceed correctly?} - Yes, the system provides direct feedback—typed text appears, date pickers open, and the “Generate Trip Plan” button clearly indicates the next step.
\end{enumerate}


\subsection*{Task 2: Research and Plan Accommodations}

\begin{figure}[H]
    \centering
    \includegraphics[width=\textwidth]{images/accommodations.png}
    \caption{Accommodations page}
    \label{fig:unique-label}
\end{figure}

\textbf{Starting point:} User clicks "Find Accommodation" from the Trip Planner.

\textbf{Action: User explores accommodation cards}  
The user browses the grid of hotel, resort, and lodge cards. The system displays each card with image, name, price, rating, and basic amenities. Selecting a card expands it to show more details. Filters and category tabs are visible but secondary.

\textbf{Cognitive Walkthrough Questions:}
\begin{enumerate}
    \item \textbf{Will the user be trying to produce the effect?} - Yes, the user wants to explore lodging options and compare amenities.
    \item \textbf{Will the user see the correct control?} - Yes, the cards are visually prominent and clearly clickable.
    \item \textbf{Will the user see that the control produces the desired effect?} - Yes, clicking a card expands details, confirming the interaction works.
    \item \textbf{Is there another control that the user might select instead of the correct one?} - Possibly, filters or tabs, but they support secondary actions and do not compete visually with the main cards.
    \item \textbf{Will the user understand the feedback to proceed correctly?} - Yes, expanded details (amenities, pricing, ratings) appear immediately, confirming the action was successful.
\end{enumerate}


\subsection*{Task 3: Discover and Explore Activities}

\begin{figure}[H]
    \centering
    \includegraphics[width=\textwidth]{images/activities.png}
    \caption{Activities page}
    \label{fig:unique-label}
\end{figure}

\textbf{Starting point:} User clicks "Discover Activities" from the Trip Planner or opens the Activities section from navigation.

\textbf{User Action: Exploring Activities}  
The user browses activity cards and interacts with category tabs (e.g., Cultural, Adventure, Food). Clicking a tab filters the activities instantly, and selecting an activity card expands it to show detailed information such as duration, price, location, and rating. Other tabs and cards are visible but do not compete with the main browsing task.

\textbf{Cognitive Walkthrough Questions:}
\begin{enumerate}
    \item \textbf{Will the user be trying to produce the effect?} - Yes, the user intends to discover available activities, so interacting with tabs and cards directly supports this goal.
    \item \textbf{Will the user see the correct control?} - Yes, the category tabs are clearly placed at the top and the activity cards occupy the main content area, making them easy to notice and understand.
    \item \textbf{Will the user see that the control produces the desired effect?} - Yes, filtered activity lists appear immediately when tabs are clicked, and the card expansion visually communicates success.
    \item \textbf{Is there another control that the user might select instead of the correct one?} - No, while other tabs exist, they support optional categories and do not conflict with the user’s primary task of browsing activities.
    \item \textbf{Will the user understand the feedback to proceed correctly?} - Yes, the interface updates instantly with filtered results and expanded details, clearly indicating the effect of the user’s actions.
\end{enumerate}


\subsection*{Task 4: Plan Wardrobe Using Weather Information}

\begin{figure}[H]
    \centering
    \includegraphics[width=\textwidth]{images/wardrobe.png}
    \caption{Wardrobe Planner page}
    \label{fig:unique-label}
\end{figure}

\textbf{Starting point:}: User opens the "Wardrobe Planner" from either the Trip Planner action cards or the main navigation.

\textbf{User Action: Selecting Packing Scenarios}  
The user selects a scenario button (e.g., Beach/Summer, Adventure/Outdoor) to see recommended clothing for the trip type and weather. The system displays a checklist with item quantities and highlights essential items. Other scenario buttons are visible but do not interfere with the selected scenario, allowing the user to easily explore multiple packing contexts.

\textbf{Cognitive Walkthrough Questions:}
\begin{enumerate}
    \item \textbf{Will the user be trying to produce the effect?} - Yes, the user intends to determine what to pack, so choosing a scenario and viewing its checklist aligns perfectly with this goal.
    \item \textbf{Will the user see the correct control?} - Yes, scenario buttons are prominent, clearly labeled, and located at the top of the interface, making them easy to identify.
    \item \textbf{Will the user see that the control produces the desired effect?} - Yes, clicking a scenario updates the displayed checklist immediately, clearly signaling the effect of the action.
    \item \textbf{Is there another control that the user might select instead of the correct one?} - Not likely; other scenario buttons represent alternative contexts but do not distract from the current selection.
    \item \textbf{Will the user understand the feedback to proceed correctly?} - Yes, the checklist with item quantities and essential markers appears instantly, making it clear that the selection was successful and guiding further packing decisions.
\end{enumerate}


\subsection*{Task 5: Create and Organize Packing Lists}

\begin{figure}[H]
    \centering
    \includegraphics[width=\textwidth]{images/packing-list1.png}
    \caption{Packing List overview with category tabs}
\end{figure}

\begin{figure}[H]
    \centering
    \includegraphics[width=\textwidth]{images/packing-list2.png}
    \caption{Packing List with checked items and progress indicators}
\end{figure}

\textbf{Starting point:} User navigates to Packing List from Trip Planner action cards.

\textbf{User Action: Organizing and Tracking Items}  
The user selects a category tab (e.g., Clothing, Electronics, Toiletries) and checks off items as they plan their packing. Progress indicators update to show completion percentage for each category. Users can navigate across tabs to compare categories or review items before packing.

\textbf{Cognitive Walkthrough Questions:}
\begin{enumerate}
    \item \textbf{Will the user be trying to produce the effect?} - Yes, the user’s goal is to organize packing efficiently and ensure nothing is forgotten, so checking off items and reviewing categories directly supports this goal.
    \item \textbf{Will the user see the correct control?} - Yes, the category tabs and checklists are prominent, clearly labeled, and positioned at the top and main area of the interface.
    \item \textbf{Will the user see that the control produces the desired effect?} - Yes, checking an item immediately shows a checkmark and updates the progress indicator, providing clear visual feedback.
    \item \textbf{Is there another control that the user might select instead of the correct one?} - Unlikely, while navigation or other options exist, they do not interfere with the main task of checking items and tracking progress.
    \item \textbf{Will the user understand the feedback to proceed correctly?} - Yes, the combination of checkmarks and progress bars clearly indicates which items have been completed and what remains, guiding the user through the packing process.
\end{enumerate}


\subsection*{Task 6: View and Plan Daily Itinerary}

\begin{figure}[H]
    \centering
    \includegraphics[width=\textwidth]{images/wardrobe.png}
    \caption{Home page}
    \label{fig:unique-label}
\end{figure}

\textbf{Starting point:} User opens the "Wardrobe Planner" from either the Trip Planner action cards or the main navigation.

\textbf{Step 1: View itinerary overview}  
The user sees the Itinerary interface with day navigation tabs and an activity timeline.  

\begin{enumerate}
    \item \textbf{Will the user be trying to produce the effect?} - Yes, the user wants to plan daily schedules, so viewing the interface matches their goal.
    \item \textbf{Will the user see the correct control?} - Yes, day tabs and timelines are prominent at the top and center of the interface.
    \item \textbf{Will the user see that the control produces the desired effect?} - Yes, selecting a day immediately updates the timeline with activities.
    \item \textbf{Is there another control that the user might select instead of the correct one?} - Unlikely; other options (e.g., navigation links) support different tasks and do not interfere.
    \item \textbf{Will the user understand the feedback to proceed correctly?} - Yes, activity times, locations, and types appear clearly on the timeline after selecting a day.
\end{enumerate}

\textbf{Step 2: Review daily activities}  
The user clicks on individual day tabs (e.g., "Day 1") and examines scheduled activities such as breakfast, city tour, lunch, museum visit, and sunset at the beach.  

\begin{enumerate}
    \item \textbf{Will the user be trying to produce the effect?} - Yes, the user is trying to understand the planned schedule for that day.
    \item \textbf{Will the user see the correct control?} - Yes, the timeline clearly lists each activity in chronological order.
    \item \textbf{Will the user see that the control produces the desired effect?} - Yes, selecting a day immediately displays all activities and details.
    \item \textbf{Is there another control that the user might select instead of the correct one?} - Minimal chance; the timeline and day tabs are clearly associated with viewing the itinerary.
    \item \textbf{Will the user understand the feedback to proceed correctly?} - Yes, the timeline visually represents activity order and type, confirming the schedule.
\end{enumerate}

\textbf{Step 3: Compare days and plan own schedule}  
The user clicks other day tabs to see different schedules and uses the color-coded activity types and icons to mentally plan their own daily activities.  

\begin{enumerate}
    \item \textbf{Will the user be trying to produce the effect?} - Yes, comparing days supports planning their personalized itinerary.
    \item \textbf{Will the user see the correct control?} - Yes, day tabs and colored icons make each day’s schedule visible and distinguishable.
    \item \textbf{Will the user see that the control produces the desired effect?} - Yes, selecting different days immediately updates the activity timeline.
    \item \textbf{Is there another control that the user might select instead of the correct one?} - No; alternate navigation does not interfere with this task.
    \item \textbf{Will the user understand the feedback to proceed correctly?} - Yes, the timeline and icons provide clear feedback on the schedule for each day.
\end{enumerate}


\subsection*{Task 7: Calculate and Track Travel Budget}

\begin{figure}[H]
    \centering
    \includegraphics[width=\textwidth]{images/budget.png}
    \caption{Budget Calculator interface with tabs and template cards}
\end{figure}

\textbf{Starting point:} User accesses the "Budget Calculator" from Trip Planner action cards or main navigation.

\textbf{Step 1: View budget templates}  
The user sees the Budget Calculator interface with tabs for Overview, Expenses, Analytics, and Templates.  

\begin{enumerate}
    \item \textbf{Will the user be trying to produce the effect?} - Yes, the user wants to explore typical travel costs, so viewing templates aligns with their goal.
    \item \textbf{Will the user see the correct control?} - Yes, the Templates tab is clearly labeled and positioned prominently at the top of the interface.
    \item \textbf{Will the user see that the control produces the desired effect?} - Yes, clicking the tab immediately displays three template cards with travel styles and total costs.
    \item \textbf{Is there another control that the user might select instead of the correct one?} - Minimal; other tabs are visible but serve different functions and do not conflict with the immediate task.
    \item \textbf{Will the user understand the feedback to proceed correctly?} - Yes, the appearance of the template cards and their cost totals clearly confirm that the tab selection succeeded.
\end{enumerate}

\textbf{Step 2: Explore a budget template}  
The user clicks on the “Mid-Range Explorer” template card to view the category breakdown (Flights, Accommodation, Food \& Dining, etc.) with visual progress bars.  

\begin{enumerate}
    \item \textbf{Will the user be trying to produce the effect?} - Yes, they want to see a detailed example of budget distribution.
    \item \textbf{Will the user see the correct control?} - Yes, the template cards are visually prominent and clearly clickable.
    \item \textbf{Will the user see that the control produces the desired effect?} - Yes, clicking the card expands to show category-level amounts and visual progress bars.
    \item \textbf{Is there another control that the user might select instead of the correct one?} - Low likelihood; other templates are options, but selecting one still aligns with the user’s goal.
    \item \textbf{Will the user understand the feedback to proceed correctly?} - Yes, the expanded breakdown clearly shows proportional spending, confirming the system responded correctly.
\end{enumerate}

\textbf{Step 3: Compare templates}  
The user clicks on “Budget Traveler” to compare with the Mid-Range Explorer template.  

\begin{enumerate}
    \item \textbf{Will the user be trying to produce the effect?} - Yes, the user wants to understand differences in typical travel costs.
    \item \textbf{Will the user see the correct control?} - Yes, template cards are arranged for easy comparison.
    \item \textbf{Will the user see that the control produces the desired effect?} - Yes, selecting another card immediately updates the category breakdown and totals.
    \item \textbf{Is there another control that the user might select instead of the correct one?} - No; templates are the main interactive elements for this task.
    \item \textbf{Will the user understand the feedback to proceed correctly?} - Yes, visual progress bars and updated totals confirm the action succeeded.
\end{enumerate}

\textbf{Step 4: Review budget tracking interface}  
The user clicks the “Overview” tab to see a summary of categories and current spending status.  

\begin{enumerate}
    \item \textbf{Will the user be trying to produce the effect?} - Yes, they want to understand overall expense tracking for planning.
    \item \textbf{Will the user see the correct control?} - Yes, the Overview tab is prominently displayed among other tabs.
    \item \textbf{Will the user see that the control produces the desired effect?} - Yes, clicking the tab updates the interface to show category summaries and tracking indicators.
    \item \textbf{Is there another control that the user might select instead of the correct one?} - Minimal; other tabs are available but serve different purposes.
    \item \textbf{Will the user understand the feedback to proceed correctly?} - Yes, totals, categories, and visual indicators clearly reflect current spending status.
\end{enumerate}


\subsection*{Task 8: Access Emergency Information and Contacts}

\begin{figure}[H]
    \centering
    \includegraphics[width=\textwidth]{images/emergency.png}
    \caption{Emergency Contacts interface with service cards and personal contact management}
\end{figure}

\textbf{Starting point:} User navigates to "Emergency Contacts" from Trip Planner action cards or main navigation.

\textbf{Step 1: Review emergency service information}  
The user sees emergency service cards (police, fire, medical) with phone numbers and contact details.  

\begin{enumerate}
    \item \textbf{Will the user be trying to produce the effect?} - Yes, the user wants to gather critical emergency information, so reviewing the service cards directly supports their goal.
    \item \textbf{Will the user see the correct control?} - Yes, the cards are prominently displayed at the top of the interface and clearly labeled by service type.
    \item \textbf{Will the user see that the control produces the desired effect?} - Yes, the visible phone numbers and labels make it immediately clear that the information is accessible.
    \item \textbf{Is there another control that the user might select instead of the correct one?} - Not likely; other controls such as adding personal contacts are secondary and do not conflict with reviewing service information.
    \item \textbf{Will the user understand the feedback to proceed correctly?} - Yes, the static display of service details provides immediate confirmation that the information is visible and available for reference.
\end{enumerate}

\textbf{Step 2: Add a personal emergency contact}  
The user clicks the “Add Contact” button, fills in name, relationship, phone number, and email. The system saves the new contact and displays it under the appropriate category (family, medical, travel).  

\begin{enumerate}
    \item \textbf{Will the user be trying to produce the effect?} - Yes, the user intends to include their own contacts for comprehensive emergency preparedness.
    \item \textbf{Will the user see the correct control?} - Yes, the “Add Contact” button is visually distinct and labeled clearly at the top or in the relevant category section.
    \item \textbf{Will the user see that the control produces the desired effect?} - Yes, completing the form results in the contact being added to the categorized list, confirming success.
    \item \textbf{Is there another control that the user might select instead of the correct one?} - Low likelihood; other controls may exist for navigation or viewing, but none conflict with adding a contact.
    \item \textbf{Will the user understand the feedback to proceed correctly?} - Yes, the system immediately shows the new contact in the list, providing clear visual confirmation.
\end{enumerate}

\textbf{Step 3: Reference contacts for planning}  
The user reviews both system-provided emergency numbers and personal contacts to ensure they have all necessary information for safe travel.  

\begin{enumerate}
    \item \textbf{Will the user be trying to produce the effect?} - Yes, the user’s goal is to consolidate and understand all emergency resources.
    \item \textbf{Will the user see the correct control?} - Yes, categorized lists and cards make browsing and referencing straightforward.
    \item \textbf{Will the user see that the control produces the desired effect?} - Yes, the static presentation confirms all entries are visible and accessible.
    \item \textbf{Is there another control that the user might select instead of the correct one?} - Not likely; reviewing contacts is the intended task and no other control competes with it.
    \item \textbf{Will the user understand the feedback to proceed correctly?} - Yes, the organized layout provides immediate understanding of which contacts are available and where to find them.
\end{enumerate}

\newpage
\section*{Outcome of the Evaluation and Planned Improvements}

The cognitive walkthrough of the Personal Travel Assistant prototype demonstrates that the interface generally supports the user's intentions effectively. For the 8 core tasks analyzed, the design successfully employs standard UI patterns—such as clear input fields, card-based layouts, and immediate visual feedback.

\subsection*{Summary of Findings}
The evaluation highlighted several key strengths and a few areas for refinement:
\begin{itemize}
    \item \textbf{Visibility of Controls:} In almost all scenarios, the correct control (buttons, inputs, tabs) was the most prominent element on the screen, minimizing search time.
    \item \textbf{Immediate Feedback:} The system excels at providing real-time response (e.g., progress bars, checklist updates), which answers the user's question, "Did my action work?"
    \item \textbf{Navigation Risks:} In Task 1 (Trip Planner) and Task 7 (Budget), the presence of secondary navigation or alternative action cards creates a minor risk of distraction, though it does not strictly prevent task completion.
\end{itemize}

\subsection*{Planned Improvements}
Based on the analysis of the cognitive walkthrough questions, the following improvements are planned to enhance usability and reduce cognitive load:

\begin{enumerate}
    \item \textbf{Strengthen Visual Hierarchy in the Trip Planner (Task 1):}
    \textit{Issue:} The Action Cards (Accommodation, Activities) visually compete with the main Trip Details form.
    \textit{Improvement:} Emphasize the Trip Details area with a clearer visual container (e.g., stronger contrast or subtle shadow) so users immediately understand it is the primary interaction zone. Action Cards can be slightly toned down to indicate they become relevant after the initial trip setup.

    \item \textbf{Increase Discoverability of Activity Management Controls (Task 6):}  
    \textit{Issue:} Users might not immediately notice how to add or edit activities for a specific day.  
    \textit{Improvement:} Introduce a persistent, clearly labeled “Add Activity” button within each day’s section, ensuring users can always see the next possible action without searching the interface.

\end{enumerate}


\newpage
\begin{thebibliography}{9} 
\bibitem{nielsen1994usability} J. Nielsen, \textit{Usability Engineering}. Academic Press, 1994. 
\bibitem{iso9241} International Organization for Standardization, \textit{ISO 9241-210:2019 Ergonomics of Human-System Interaction — Human-centred Design for Interactive Systems}, 2019. 
\bibitem{card1983psychology} S. Card, T. Moran, A. Newell, \textit{The Psychology of Human-Computer Interaction}. Erlbaum Associates, 1983. 
\bibitem{wharton1994cognitive} C. Wharton, J. Rieman, C. Lewis, P. Polson, \textit{The Cognitive Walkthrough Method: A Practitioner's Guide}. In J. Nielsen \& R. Mack (Eds.), \textit{Usability Inspection Methods}, John Wiley \& Sons, 1994.
\end{thebibliography}

\end{document}
