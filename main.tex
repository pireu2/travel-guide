\documentclass[11pt,a4paper]{article}

\usepackage[margin=1in]{geometry}
\usepackage[utf8]{inputenc}
\usepackage[T1]{fontenc}
\usepackage{lmodern}
\usepackage{xcolor}
\usepackage{hyperref}
\usepackage{enumitem}
\usepackage{booktabs}
\usepackage{tabularx}
\usepackage{titlesec}

\definecolor{accent}{HTML}{006666}
\hypersetup{
  colorlinks=true,
  linkcolor=accent,
  urlcolor=accent
}

\titleformat{\section}{\large\bfseries\color{accent}}{\thesection}{0.6em}{}
\titleformat{\subsection}{\bfseries}{\thesubsection}{0.5em}{}

\title{\vspace{-1em}\textbf{Project Documentation}\\
\large Personal Travel Assistant}
\author{Team: Group 1\\
Members: Simion Oltean Callian, Duica Sebastian, Denisa Sarafinceanu, Cosmin Savut}
\date{October 2025 – January 2026}

\begin{document}
\maketitle
\vspace{-1em}

\tableofcontents
\newpage

% =========================
\refstepcounter{section}
\section*{Phase 2 – Project Proposal Presentation}

\subsection{Description of the Project Problem Domain}

Travel planning entails transport, accommodation, and day activity booking all over
several online services. Most users alternate between booking sites, map apps, and calendars,
which enhance mental workload and likelihood of forgetting information; reduce steps and offering
clear feedback eliminates this task \cite{card1983psychology}. Our project — the \textbf{Personal Travel Assistant} — is
to compile all travel information into one easy interface to assist users in creating, editing, and
track their travels effectively.
Travel planning involves coordinating transport, accommodation, and daily activities across several online services.

\subsection{User Types}
\begin{itemize}
  \item \textbf{Students and young travelers} – seek affordable trips, rely mainly on mobile devices, moderate tech proficiency.
  \item \textbf{Business travelers} – need fast, reliable scheduling; high device proficiency; value reminders and sync with calendars.
  \item \textbf{Families or groups} – require clarity, simple navigation, and shared itineraries.
\end{itemize}

\subsubsection*{Knowledge Levels}
\begin{itemize}
  \item \textbf{Domain knowledge:} users can have only basic knowledge of bookings and travel choices.
  \item \textbf{Device knowledge:} is minimal; the interface uses familiar patterns (buttons, icons) and short texts, compatible with phones, tablets, and computers.
\end{itemize}

\subsubsection*{Special User Requirements}

The design follows \textit{WCAG 2.2 AA} accessibility goals and relies on human-centred design prin-
ciples to ensure equitable access \cite{iso9241}

\begin{itemize}
  \item High-contrast color theme and scalable text for visually impaired users.
  \item Voice-assisted search for users with motor disabilities.
  \item Touch-friendly layout and short forms to minimize effort.
\end{itemize}

\subsection{Main Challenges and Motivation}
\begin{itemize}
  \item \textbf{Fragmented tools} – travelers must manage several unrelated apps.
  \item \textbf{Uncertain conditions} – cancellations, weather, and delays require dynamic re-planning.
  \item \textbf{Information overload} – difficult to find trustworthy local experiences.
\end{itemize}
\textbf{Motivation:} integrating essential travel information in one accessible application saves time and reduces stress.
Usability research stresses the value of simplicity, visibility of system status, and error prevention when managing time-critical tasks \cite{nielsen1994usability}.

\subsection{Context of Use}
The app could be used:
\begin{itemize}
  \item On \textbf{mobile phones} during trips — noisy, interruptive environments (airports, buses), where concise feedback and large touch targets help maintain flow \cite{nielsen1994usability}.
  \item On \textbf{desktops} for planning — quieter, analytical context suited to comparison and review.
  \item With \textbf{time-sensitive tasks} (boarding, check-ins), so quick feedback and reminders are essential.
  \item \textbf{Offline access} is needed for tickets and maps when connectivity is lost.
\end{itemize}

\subsection{Existing Applications}
\begin{tabularx}{\linewidth}{l X X}
\toprule
\textbf{App} & \textbf{Advantages} & \textbf{Limitations} \\
\midrule
TripAdvisor & Many reviews and location data & Hard to personalize; no schedule integration \\
Booking.com & Reliable booking engine & Limited itinerary organization \\
Google Maps & Excellent navigation support & Overwhelming interface for trip planning \\
TripIt & Auto-import of bookings & Premium features locked; limited local discovery \\
\bottomrule
\end{tabularx}

\subsection{Interconnection with Other Tools}
The Personal Travel Assistant integrates existing services to avoid manual re-entry of data:
\begin{itemize}
  \item \textbf{Google Maps API} – display routes and travel times.
  \item \textbf{Email parsing} – detect booking confirmations and import details.
\end{itemize}
Consistent interaction patterns and clearly scoped integrations support efficiency and satisfaction in line with human-centred design guidance \cite{iso9241}.

% =========================
\newpage
\refstepcounter{section}
\section*{Phase 2a – Task Analysis Examples}

This phase presents two simple travel related tasks that typical users of the Personal Travel Assistant project would perform.

\subsection{Task 1 — Plan a Weekend Trip}
\textbf{Goal:} create a basic travel plan for a short trip.

\begin{enumerate}
  \item Open the app and tap “Plan new trip.”
  \item Enter the destination and dates.
  \item Select one suggested hotel.
  \item Choose one transport option (train or flight).
  \item Save the trip.
\end{enumerate}

\textbf{Expected result:} the trip is saved, and the user can see it on the main screen with hotel and travel details.

\subsection{Task 2 — Change the Travel Date}
\textbf{Goal:} adjust a planned trip when the departure date changes.

\begin{enumerate}
  \item Open “My Trips” and select the saved trip.
  \item Tap “Edit” and choose a new start date.
  \item The app updates the return date automatically.
  \item Confirm and save the changes.
\end{enumerate}

\textbf{Expected result:} the trip shows the new dates, and reminders update automatically.

% =========================
\newpage
\refstepcounter{section}
\section*{Phase 3 – Task Analysis}

This phase describes 12 simple and realistic tasks that users can perform with the \textbf{Personal Travel Assistant}.  
Each task includes the starting point, type of user, actions performed, motivation, and the context in which it occurs.

\subsection*{Task 1 – Search for a Destination}
\textbf{Starting point:} Home screen \\
\textbf{Users:} All users \\
\textbf{What:} The user types the name of a city or country into the search bar to see available information.  
The system shows details such as basic facts, attractions, and weather. \\
\textbf{Why:} The user wants to explore or gather information about a place before planning a trip. \\
\textbf{Context:} Usually done from home on a mobile device or laptop while browsing casually.

\subsection*{Task 2 – View Destination Details}
\textbf{Starting point:} Search results page \\
\textbf{Users:} All users \\
\textbf{What:} The user selects one destination from the list to view more details such as images, short descriptions, and nearby points of interest. \\
\textbf{Why:} To better understand what the destination offers before deciding to plan a trip. \\
\textbf{Context:} Done during free time, either on desktop or mobile.

\subsection*{Task 3 – Create a New Trip}
\textbf{Starting point:} Dashboard or Home screen \\
\textbf{Users:} All users \\
\textbf{What:} The user clicks “Add Trip,” enters the destination name, start and end dates, and an optional description.  
The system saves this as a new trip in the user’s list. \\
\textbf{Why:} To start organizing an upcoming journey or vacation. \\
\textbf{Context:} Usually done at home in a calm environment, on mobile or desktop.

\subsection*{Task 4 – Add Notes to a Trip}
\textbf{Starting point:} Trip details page \\
\textbf{Users:} All users \\
\textbf{What:} The user adds short notes such as reminders, packing lists, or travel tips inside the trip view.  
The notes are saved automatically. \\
\textbf{Why:} To keep all relevant information about the trip in one place. \\
\textbf{Context:} Done while planning or during the trip when new ideas come up.

\subsection*{Task 5 – Add an Activity}
\textbf{Starting point:} Trip details page \\
\textbf{Users:} All users \\
\textbf{What:} The user adds one or more planned activities, such as visiting a museum, going on a hike, or attending an event, with optional time and description. \\
\textbf{Why:} To organize each day’s schedule and make the trip easier to follow. \\
\textbf{Context:} Usually done during trip planning, on desktop or mobile.

\subsection*{Task 6 – Edit a Trip}
\textbf{Starting point:} My Trips section \\
\textbf{Users:} All users \\
\textbf{What:} The user selects a trip and modifies details such as travel dates, notes, or activities.  
The system updates the information and saves changes automatically. \\
\textbf{Why:} Travel dates or plans may change, and the user wants to keep the app up to date. \\
\textbf{Context:} Done quickly from mobile or laptop, before or during travel.

\subsection*{Task 7 – Delete a Trip}
\textbf{Starting point:} My Trips section \\
\textbf{Users:} All users \\
\textbf{What:} The user deletes a trip that is no longer needed or has already been completed.  
The app asks for confirmation before removing it permanently. \\
\textbf{Why:} To keep the trip list organized and remove old entries. \\
\textbf{Context:} Short action done at home or while organizing previous travel records.

\subsection*{Task 8 – View Saved Trips}
\textbf{Starting point:} Dashboard or My Trips page \\
\textbf{Users:} All users \\
\textbf{What:} The user opens the list of all saved trips to review them.  
Selecting a trip opens its details such as dates, notes, and activities. \\
\textbf{Why:} To see progress, check upcoming trips, or revisit past ones. \\
\textbf{Context:} Done casually from desktop or mobile, often while planning.

\subsection*{Task 9 – Share a Trip}
\textbf{Starting point:} Trip details page \\
\textbf{Users:} Friends, families, groups \\
\textbf{What:} The user shares the trip summary or itinerary via link, email, or messaging app.  
The recipient can view basic trip information but cannot edit it. \\
\textbf{Why:} To share travel ideas or coordinate with companions. \\
\textbf{Context:} Usually done online from a mobile device.

\subsection*{Task 10 – Mark a Trip as Completed}
\textbf{Starting point:} My Trips section \\
\textbf{Users:} All users \\
\textbf{What:} The user marks a finished trip as completed, which moves it from the “Upcoming” to “Past Trips” list. \\
\textbf{Why:} To separate past experiences from future travel plans and keep data organized. \\
\textbf{Context:} Done after returning home from the trip, on desktop or mobile.

\subsection*{Task 11 – Set a Reminder}
\textbf{Starting point:} Trip details page \\
\textbf{Users:} All users \\
\textbf{What:} The user creates a simple reminder for a specific date or time, such as the departure day or activity start time.  
The app sends a notification when the event is near. \\
\textbf{Why:} To avoid missing important travel activities or departure times. \\
\textbf{Context:} Usually set before the trip, on mobile devices.

\subsection*{Task 12 – Change App Settings}
\textbf{Starting point:} Settings page \\
\textbf{Users:} All users \\
\textbf{What:} The user adjusts preferences such as app language, theme (light/dark), or notification options.  
Settings are saved automatically and applied globally. \\
\textbf{Why:} To personalize the app according to personal comfort and needs. \\
\textbf{Context:} Done occasionally, at any time, on mobile or desktop.


\newpage
\begin{thebibliography}{9} \bibitem{nielsen1994usability} J. Nielsen, \textit{Usability Engineering}. Academic Press, 1994. \bibitem{iso9241} International Organization for Standardization, \textit{ISO 9241-210:2019 Ergonomics of Human-System Interaction — Human-centred Design for Interactive Systems}, 2019. \bibitem{card1983psychology} S. Card, T. Moran, A. Newell, \textit{The Psychology of Human-Computer Interaction}. Erlbaum Associates, 1983. \end{thebibliography}

\end{document}
